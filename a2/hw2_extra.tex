\documentclass[11pt]{article}

\renewcommand{\today}{February 13, 2001}

\usepackage{fullpage}
\usepackage{fancyhdr}

\lhead{Algorithms}
\chead{}
\rhead{\today}
\lfoot{}
\cfoot{\thepage}
\rfoot{}

\pagestyle{fancy}

\newcommand{\op}[1]{{\sf #1}}

\begin{document}

\begin{center}
  \bf {\large Homework 3}
\end{center}

\begin{itemize}

\item[\bf 1:] {\bf Minimum Spanning Trees}
  
  The following algorithm can be shown to compute a minimum spanning
  tree (MST) of a connected undirected graph $G=(V,E)$, where each
  edge has a real valued weight.  Initially let each node of $V$ be a
  tree of size one.  Let the set of trees be denoted $\cal{T}$.
  Repeatedly pick an arbitrary tree $t\in \cal{T}$ and grow $t$ by
  joining $t$ with a tree $t'\in \cal{T}$ that is connected to $t$ by
  an edge $e$ of minimum weight. The edge $e$ will be part of the
  computed MST. This generic algorithm forms the basis for several
  efficient algorithms for computing a MST.

  \begin{itemize} 
  \item[a)] Adapt the algorithm such that it for a planar graph $G=(V,E)$ finds a
    MST in time~$O(|V|)$. The algorithm should proceed in $O(\log
    |V|)$ phases.  After each phase the algorithm should contract each
    subtree to a single node and remove multiple edges and self loops.
  \end{itemize}
  
  Let $G=(V,E)$ be a connected undirected graph, with $n=|V|$ and
  $m=|E|$, where each edge has  a real valued weight. In the
  following we consider a variant of the generic algorithm that works
  in a sequence of phases. In each phase small subtrees of a MST are
  identified. After a phase the subtrees found in the phase are
  contracted to single nodes.  For each phase we have two parameters $s_i$
  and $k_i$, for $i\geq 0$. The number of trees at the start of
  phase $i$ is denoted~$s_i$, where $s_0=n$. The parameter
  $k_i=2^{2m/s_i}$ is denoted the heap limit of phase~$i$.
  
  In phase $i$, initially all nodes are subtrees of size one and are
  unmarked. An unmarked subtree~$t$ is grown until one of the
  following three conditions is true, in which case $t$ is marked.
  \begin{itemize}
  \item $t$ is adjacent to $>k_i$ nodes outside $t$.
  \item $t$ has been joined with a marked tree, or
  \item $t$ is not adjacent to another tree,
  \end{itemize}
  A phase ends when all nodes have been marked.

  \begin{itemize}
  \item[b)] Show how each phase of the algorithm can be implemented to
    take time $O(s_i\log k_i+m)$ using Fibonacci heaps.
  \item[c)] Show that the number of trees after a phase is at most
    $2{m_i}/{k_i}$, and that the heap limit $k_{i+1}\geq 2^{k_i}$.
  \item[d)] Argue that there is at most $O(\log^* n)$ iterations, and
    that the running time of the algorithm becomes $O(m\log^* n)$,
    where $\log^* n=\min \{ i \mid \log^{(i)} n\leq 1 \}$, $\log^{(1)}=\log n$ 
    and $\log^{(i+1)} n=\log\log^{(i)} n$, for $i>1$.
  \end{itemize}

\item[\bf 2:] {\bf Augmenting Paths of Maximum Bottleneck Capacity}
  
  Show how to calculate in time $O(|E|+|V|\log|V|)$ an augmenting path of maximum
  bottleneck capacity in a graph $G=(V,E)$ with capacity $c$.
  
\item[\bf 3:] {\bf Ghostbusters}

  A group of $n$ Ghostbusters is battling $n$ ghosts. Each
  Ghostbuster is armed with a proton pack, which shoots a stream at a
  ghost, eradicating it. A stream goes in a straight line and
  terminates when it hits the ghost. The Ghostbusters decide upon the
  following strategy. They will pair off with the ghosts, forming $n$
  Ghostbuster-ghost pairs, and then simultaneously each Ghostbuster
  will shoot a stream at his or her chosen ghost. As we all know, it
  is \emph{very} dangerous to let streams cross, and so the
  Ghostbusters must choose pairings for which no streams will cross.
  
  Assume that the position of each Ghostbuster and each ghost is a
  fixed point in the plane and that no three positions are on the same
  line.
  \begin{itemize}
  \item[a)] Argue that there exists a line passing through one
    Ghostbuster and one ghost such the number of Ghostbusters on one
    side of the line equals the number of ghosts on the same side.
    Describe how to find such a line in $O(n\log n)$ time.
  \item[b)] Give an $O(n^2\log n)$ time algorithm to pair Ghostbusters
    with ghosts in such a way that no streams cross.
  \end{itemize}


\item[\bf 4:] {\bf Self Adjusting Trees $^*$}
  
  In this exercise we consider search trees that do not maintain any
  explicit balance information. Each node is represented by a record
  consisting of three fields: an element from some totally ordered
  universe, a pointer to the left child, and a pointer to the right
  child (non existing children are represented by \textsf{nil}). The
  elements are maintained in inorder. The crucial operation on a tree
  is the following operation:

  \begin{itemize}
  \item \textsf{splay}$(i,T)$ reorganize $T$ such that the node
    containing element $i$ becomes the root. If $i\not\in T$ then
    either $\max\{k\in T|k<i\}$ or $\min\{k\in T|k>i\}$ becomes the
    root.
  \end{itemize}
  
  Using the \textsf{splay} operation we can now implement the
  following three operations:

  \begin{itemize}
  \item \textsf{Member}$(i,T)$ determines if $i\in T$ by performing
    \textsf{splay}$(i,T)$ and returning true if and only if the new
    root of $T$ contains $i$.
  \item \textsf{Insert}$(i,T)$ inserts element $i$ in $T$ as follows.
    First perform \textsf{splay}$(i,T)$. Assume the new root $v$
    contains $\max\{k\in T|k<i\}$ (the case $\min\{k\in T|k>i\}$ is
    handled symmetrically). Create a new root with element $i$, left
    child $v$, and right child being the right child of $v$. The right
    child of $v$ becomes \textsf{nil} and the left child of $v$
    remains unchanged.
  \item \textsf{Delete}$(i,T)$ deletes $i\in T$ as follows. First
    perform \textsf{splay}$(i,T)$ such that the root contains~$i$. Let
    $T_L$ and $T_R$ be respectively the subtree rooted at the left child and
    the right child of the root.  Remove the root and let $T=T_L$. Perform
    \textsf{splay}$(i,T)$, i.e.\ the right child of the root of $T$ is
    \textsf{nil}, and make $T_R$ the right child of the root of $T$.
  \end{itemize}
  
  The \textsf{splay} operation is implemented by first performing an
  ordinary search in $T$ for $i$. The node found is then moved to the
  root by performing a sequence of rotations:

  \begin{center}
    \unitlength0.4pt
    \begin{picture}(600,185)(0,35)
      \newcommand{\TREE}[2]{
        \put(#1){
          \put(0,0){\line(-1,-2){35}}
          \put(0,0){\line(1,-2){35}}
          \put(-35,-70){\line(1,0){70}}
          \put(0,-55){\makebox[0mm][c]{#2}}
          }}
      \newcommand{\NODE}[1]{\put(#1){\circle*{10}}}

      \put(-20,0){
        \TREE{50,100}{$A$}
        \TREE{160,100}{$B$}
        \TREE{215,155}{$C$}
        \NODE{105,155} \put(95,155){\makebox[0mm][r]{$x$}}
        \put(105,155){\line(-1,-1){55}}
        \put(105,155){\line( 1,-1){55}}
        \NODE{160,210} \put(150,210){\makebox[0mm][r]{$y$}}
        \put(160,210){\line(-1,-1){55}}
        \put(160,210){\line( 1,-1){55}}
        }

      \put(300,125){\makebox[0mm][c]{\textsf{rotate}$(x)$}}
      \put(250,110){\vector(1,0){100}}
      \put(350,100){\vector(-1,0){100}}
      \put(300,75){\makebox[0mm][c]{\textsf{rotate}$(y)$}}

      \put(300,0){
        \TREE{105,155}{$A$}
        \TREE{160,100}{$B$}
        \TREE{270,100}{$C$}
        \NODE{215,155} \put(225,155){\makebox[0mm][l]{$y$}}
        \put(215,155){\line(-1,-1){55}}
        \put(215,155){\line( 1,-1){55}}
        \NODE{160,210} \put(170,210){\makebox[0mm][l]{$x$}}
        \put(160,210){\line(-1,-1){55}}
        \put(160,210){\line( 1,-1){55}}
      }
    \end{picture}
  \end{center}
  
  More specifically, let $x$ be the node found. The node $x$ is move to the root by
  iterating the following steps:

  \begin{itemize}
  \item[(\textit{i})] If $x$ has a parent but no grandparent, apply
    \textsf{rotate}$(x)$.
  \item[(\textit{ii})] If $x$ has a parent $y$ and a grandparent, and
    both $x$ and $y$ are left children (or both right children), then apply
    \textsf{rotate}$(y)$ followed by \textsf{rotate}$(x)$.
  \item[(\textit{iii})] If $x$ has a parent $y$ and a grandparent, and
    $x$ is a left child and $y$ a right child (or $x$ right child and
    $y$ a left child), then apply \textsf{rotate}$(x)$ followed by
    \textsf{rotate}$(x)$.
  \end{itemize}
  
  The example below illustrates the effect of applying
  \textsf{splay}$(1,T)$ to the tree to the left.

  \begin{center}
    \small
    \unitlength1.3pt
    \begin{picture}(200,75)(-10,0)
      \newcommand{\TREE}[2]{
        \put(#1){
          \put(0,0){\line(-1,-2){35}}
          \put(0,0){\line(1,-2){35}}
          \put(-35,-70){\line(1,0){70}}
          \put(0,-55){\makebox[0mm][c]{#2}}
          }}
      \newcommand{\NODE}[1]{\put(#1){\circle*{3}}}

      \NODE{ 0, 0} \put(-10,0){1} \put( 0, 0){\line(1,1){10}}
      \NODE{10,10} \put( 0,10){2} \put(10,10){\line(1,1){10}}
      \NODE{20,20} \put(10,20){3} \put(20,20){\line(1,1){10}}
      \NODE{30,30} \put(20,30){4} \put(30,30){\line(1,1){10}}
      \NODE{40,40} \put(30,40){5} \put(40,40){\line(1,1){10}}
      \NODE{50,50} \put(40,50){6} \put(50,50){\line(1,1){10}}
      \NODE{60,60} \put(50,60){7} \put(60,60){\line(1,1){10}}
      \NODE{70,70} \put(60,70){8}

      \put(95,45){\makebox[0mm][c]{\normalsize\textsf{splay}$(1,T)$}}
      \put(75,40){\vector(1,0){40}}

      \put(100,0){
        \NODE{40,20} \put(45,20){3} \put(40,20){\line(-1,1){10}}
        \NODE{30,30} \put(20,30){2} \put(30,30){\line(1,1){10}}
        \NODE{50,30} \put(55,30){5} \put(50,30){\line(-1,1){10}}
        \NODE{40,40} \put(30,40){4} \put(40,40){\line(1,1){10}}
        \NODE{60,40} \put(65,40){7} \put(60,40){\line(-1,1){10}}
        \NODE{50,50} \put(40,50){6} \put(50,50){\line(1,1){10}}
        \NODE{60,60} \put(65,60){8} \put(60,60){\line(-1,1){10}}
        \NODE{50,70} \put(40,70){1}
        }

    \end{picture}
  \end{center}
  
  Let $|T(v)|$ denote the number of nodes in the subtree $T(v)$ rooted
  at a node~$v$. Let $\mu(v)=\log |T(v)|$. To do an amortized analysis
  of the operations we define the potential of a tree by the function
  \[ \Phi(T) = \left\lceil \sum_{v\in T} \mu(v)\right\rceil\;. \]

  \begin{itemize}
  \item[a)] Show that each application of (\textit{ii}) and
    (\textit{iii}) to a node $x$ implies a change in potential
    $\Delta\Phi(T) \leq 3(\mu(x')-\mu(x))-1$, where $x'$ refers to
    $x$ after applying the two rotations. Prove first that
    $2\log(a+b)-\log a-\log b>1$, for all  $a,b>0$.
  \item[b)] Show that each \textsf{splay} operation takes amortized time $O(\log n)$.
  \item[c)] Show that each of the operations \textsf{Insert},
    \textsf{Delete} and \textsf{Member} takes amortized time $O(\log
    n)$.
  \end{itemize}
  
\end{itemize}

\end{document}